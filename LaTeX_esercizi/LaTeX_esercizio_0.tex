\documentclass[12pt,a4paper]{letter}

% ----------------------
% Preambolo
% ----------------------

% Pacchetti lingua e tastiera
\usepackage[utf8]{inputenc}
\usepackage[italian]{babel}

% Altri pacchetti da inserire
\usepackage[a-1b]{pdfx}
\usepackage[a4paper,top=2cm, bottom=2cm, left=3cm, right=3cm]{geometry}

% Impostazioni generali lettera
\signature{I.I. Lorenzo Guetti}

\address{
A tutti gli studenti \\
Ai genitori/responsabili \\
Ai docenti \\
Al personale A.T.A.
}

\date{22 settembre 2025}

% ----------------------
% Inizio documento
% ----------------------
\begin{document}

\begin{letter}{\textbf{Oggetto:} Regolamento interno in materia di utilizzo dei cellulari}

\opening{Si comunica di seguito quanto previsto dal Regolamento interno di Istituto in tema di utilizzo dei cellulari a scuola:}

“Durante le lezioni è vietato l’uso del telefono cellulare. Gli studenti all’inizio delle lezioni devono depositare il telefono cellulare nell’apposito cestino presente in classe, riprenderlo all’inizio della ricreazione e depositarlo di nuovo alla fine della ricreazione fino al termine delle lezioni.
Quando gli studenti si trasferiscono in altre aule o nei laboratori o in palestra devono portare con sé i telefoni cellulari e depositarli nei cestini o nei contenitori presenti.”

La violazione di tale condotta comporta l’irrogazione dell’annotazione disciplinare nel registro elettronico da parte del docente e il contestuale ritiro temporaneo del telefono cellulare, o altro dispositivo elettronico, da depositare a cura del docente nell’ufficio del collaboratore vicario del Dirigente scolastico.

Il telefono cellulare sarà restituito allo studente dopo richiesta telefonica al collaboratore vicario da parte del responsabile genitoriale.

E’ comunque facoltà del docente permettere l’uso dei dispositivi elettronici per attività funzionali
alla didattica”

\vspace{1 cm}

\closing{Cordiali saluti,}

\end{letter}
\end{document}