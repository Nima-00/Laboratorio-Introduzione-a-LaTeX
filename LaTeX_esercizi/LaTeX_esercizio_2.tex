\documentclass[a4paper,11pt]{article}

% ----------------------------
% Preambolo
% ----------------------------

\usepackage[utf8]{inputenc}     % Pacchetto tastiera
\usepackage[italian]{babel}     % Pacchetto lingua

\usepackage{amsmath}            % Pacchetti per la matematica
\usepackage{amssymb}
\usepackage{amsthm}

\usepackage{array}              % Pacchetti tabelle
\usepackage{tabularx}
\usepackage{hyperref}           % Pacchetto per iper-link
\usepackage{graphicx}           % Pacchetto per le immagini
\usepackage{xcolor}             % Pacchetto colori

% Altri campi opzionali determinati dalla classe
\title{Esercizi Settimana 2}
\author{Nome Cognome}
\date{XX-settembre-2025}


\begin{document}

    \maketitle

    \section{Apici e pedici}
    \subsection*{Energia cinetica rotazionale}
    L'\textit{energia cinetica} di un oggetto in rotazione con velocità angolare $\omega$ può essere scritta come:
    \[
    K_{rot} = \frac{1}{2} I\omega^2
    \]

    \section{Tabelle}

    \subsection*{Simboli utili}

    \begin{tabular}{|m{1cm} m{2.5cm}|m{1cm} m{2.5cm}|m{1cm} m{2.5cm}|}\hline
        $\alpha$ & \verb|\alpha|& $\beta$ &\verb|\beta|& $\gamma$ & \verb|\gamma|\\
        $\lambda$ & \verb|\lambda|& $\mu$ &\verb|\mu|& $\pi$ & \verb|\pi|\\
        $\leq$ & \verb|\leq|& $\geq$ &\verb|\geq|& $\neq$ & \verb|\neq|\\
        $\sum$ & \verb|\sum|& $\prod$ &\verb|\prod|& $\int$ & \verb|\int|\\ \hline
    \end{tabular}

    \subsection*{Orario anno scolastico 2025-2026}
    
    \begin{table}[h]
        \centering
        \begin{tabular}{|l|c|c|c|c|c|c|}
            \hline
            Orario &Lunedì & Martedì & Mercoledì & Giovedì & Venerdì & Sabato \\
            \hline
            \hline
            8.05 - 8.55 & ... & ... & ... & ... & ... & ... \\
            8.05 - 8.55 & ... & ... & ... & ... & ... & ... \\
            8.05 - 8.55 & ... & ... & ... & ... & ... & ... \\
            8.05 - 8.55 & ... & ... & ... & ... & ... & ... \\
            8.05 - 8.55 & ... & ... & ... & ... & ... & ... \\
            8.05 - 8.55 &     & ... &     &     &     &     \\
            \hline
        \end{tabular}
        \caption{Orario Classe ...}
        \label{tab:orario}
    \end{table}

    \section{Ambienti Matematici}
    \subsection*{Criteri di congruenza dei triangoli (CCT)}
    \newtheorem*{cct1}{1° CCT}
    \begin{cct1}
        Due triangoli che hanno rispettivamente congruenti due lati e l'angolo tra di essi compreso sono congruenti.
    \end{cct1}

    \newtheorem*{cct2}{2° CCT}
    \begin{cct2}
        Due triangoli che hanno rispettivamente congruenti un lato ed i due angoli ad essi adiacenti sono congruenti.
    \end{cct2}

    \newtheorem*{cct3}{3° CCT}
    \begin{cct3}
        Due triangoli che hanno i tre lati rispettivamente congruenti sono congruenti.
    \end{cct3}

    \subsection*{Teoremi e dimostrazioni}
    \newtheorem{teo}{Teorema}
    \begin{teo}
        I numeri primi sono infiniti.
    \end{teo}
    \begin{proof}
        Supponiamo, per assurdo, che l'insieme dei numeri primi sia finito e sia $P=\{p_0, p_1, ...  , p_k\}$.\\
        Consideriamo ora il numero
        \[
            m = \prod_{i=0}^k p_i +1
        \]
        Tale numero non è divisibile per alcun numero in $P$, dunque abbiamo due casi:
        \begin{enumerate}
            \item $m$ è primo, ma allora abbiamo trovato un primo che non sta in $P$.
            \item $m$ è composto, ma allora esiste un primo $q$ che divide $m$. Tuttavia $q$ non può appartenere a $P$ poichè nessun elemento di $P$ divide $m$.
        \end{enumerate}
        In entrambi i casi si raggiunge un assurdo, perciò l'insieme dei numeri primi deve essere infinito.
    \end{proof}

    \section{Allineamento di espressioni}
    \subsection*{Scomposizione di un polinomio}
    \begin{align*}
    x^3-3x^2-4x+12 &= x^2(x-3)-4(x-3)       \\
                   &= (x^2-4)(x-3)          \\
                   &= (x-2)(x+2)(x-3)       \\
    \end{align*}

    \section{Link}
    \subsection*{Inserire un link all'interno del testo}

    \textit{3Blue1Brown} è un famoso canale YouTube di divulgazione matematica che grazie alle sue animazioni riesce a dare un'intuizione visiva di formule e teoremi a prima vista molto astrusi.\\
    Il link al canale è \href{https://www.3blue1brown.com/}{{\color{blue} 3blue1brown}}. [Worth a visit!]

    \section{Riferimenti}
    \subsection*{Creare riferimenti all'interno del testo}
    La tabella \ref{tab:orario} mostra l'orario settimanale definitivo della classe ... per l'anno scolastico 2025/2026.

    \section{Immagini}
    \subsection*{Inserimento di due immagini}
    \begin{figure}[h]
        \centering
        \begin{minipage}{0.45\textwidth}
            \centering
            \includegraphics[height=5cm]{immagini/pappagallo.jpg}
            \caption{Pappagallo}
        \end{minipage}
        \hfill
        \begin{minipage}{0.45\textwidth}
            \centering
            \includegraphics[height=5cm]{immagini/albatro.jpg}
            \caption{Albatro}
        \end{minipage}
    \end{figure}

\end{document}