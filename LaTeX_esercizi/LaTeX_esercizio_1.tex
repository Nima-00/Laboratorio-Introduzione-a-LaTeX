\documentclass[a4paper,11pt]{article}

% ----------------------------
% Preambolo
% ----------------------------

% Pacchetti per la lingua
\usepackage[utf8]{inputenc}
\usepackage[italian]{babel}

% Pacchetti per la matematica
\usepackage{amsmath}
\usepackage{amssymb}
\usepackage{amsthm}


% Altri campi opzionali determinati dalla classe
\title{Esercizi Settimana 1}
\author{Nome Cognome}
\date{XX-settembre-2025}


\begin{document}

    \maketitle

    \section*{Esercizio 1: \textit{liste numerate}}
    \textit{Ordine arrivo gara di corsa studentesca}
    \begin{enumerate}
        \item Giovanni
        \item Franco
        \item Paolo
        \item Stefano
    \end{enumerate}

    \section*{Esercizio 2: \textit{liste non numerate}}
    \underline{Lista della spesa}
    \begin{itemize}
        \item Pomodori
        \item Latte
        \item Detersivo
    \end{itemize}

    \section*{Esercizio 3: \textit{modalità inline}}
    Disegnare il grafico della funzione $f(x)=x^2-1$ con $x\in[-10, 10]$ e determinare i punti di intersezioni con gli assi cartesiani.

    \section*{Esercizio 4: \textit{modalità display}}
    Disegnare il grafico della funzione 
    \[
    f(x)=x^2-1
    \qquad x\in[-10, 10]
    \]
    e determinare i punti di intersezioni con gli assi cartesiani.

    \section*{Esercizio 5: \textit{funzioni definite a tratti}}
    \[
    f(x) =
    \begin{cases}
        x^5-3 \quad x<10 \\
        x^2-4 \quad 10\le x < 14 \\
        x^3+1 \quad x\ge14
    \end{cases}
    \]

    \section*{Esercizio 6: \textit{Formule di addizione e sottrazione}}
    \begin{equation}
    cos(\alpha \pm \beta) = cos(\alpha)\ cos(\beta) \mp sen(\alpha)\ sen(\beta)
    \end{equation}
    \begin{equation}
    sen(\alpha \pm \beta) = sen(\alpha)\ cos(\beta) \pm cos(\alpha)\ sen(\beta)
    \end{equation}
    \begin{equation}
    tan(\alpha \pm \beta) = \frac{tan(\alpha) \pm tan(\beta)}{1 \mp tan(\alpha)\ tan(\beta)}
    \end{equation}

    \section*{Esercizio 7: \textit{Formule di prostaferesi}}
    \begin{equation} sen(\alpha) \pm sen(\beta) = 2sen\left(\frac{\alpha \pm \beta}{2}\right)cos\left(\frac{\alpha \mp \beta}{2}\right) \end{equation}
    \begin{equation} cos(\alpha) + cos(\beta) = 2cos\left(\frac{\alpha + \beta}{2}\right)cos\left(\frac{\alpha - \beta}{2}\right) \end{equation}
    \begin{equation} cos(\alpha) - cos(\beta) = -2sen\left(\frac{\alpha + \beta}{2}\right)sen\left(\frac{\alpha - \beta}{2}\right) \end{equation}

\end{document}